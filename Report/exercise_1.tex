%==============================================================================
\section{Άσκηση 1: Αριθμητική/Λογική Μονάδα (ALU)}
%==============================================================================

\subsection{Περιγραφή}

Η ALU είναι ένα συνδυαστικό κύκλωμα 32-bit που υλοποιεί τις ακόλουθες λειτουργίες:

\begin{table}[H]
\centering
\caption{Πίνακας λειτουργιών ALU}
\begin{tabular}{|c|l|}
\hline
\textbf{alu\_op} & \textbf{Λειτουργία} \\
\hline
1000 & Λογική AND \\
1001 & Λογική OR \\
1010 & Λογική NOR \\
1011 & Λογική NAND \\
1100 & Λογική XOR \\
0100 & Προσημασμένη Πρόσθεση \\
0101 & Προσημασμένη Αφαίρεση \\
0110 & Προσημασμένος Πολλαπλασιασμός \\
0000 & Λογική ολίσθηση δεξιά \\
0001 & Λογική ολίσθηση αριστερά \\
0010 & Αριθμητική ολίσθηση δεξιά \\
0011 & Αριθμητική ολίσθηση αριστερά \\
\hline
\end{tabular}
\end{table}

\subsection{Αρχιτεκτονική}

Η ALU σχεδιάστηκε με τα ακόλουθα χαρακτηριστικά:
\begin{itemize}
    \item Δύο είσοδοι 32-bit (\texttt{op1}, \texttt{op2}) σε αναπαράσταση συμπληρώματος ως προς 2
    \item Είσοδος επιλογής λειτουργίας 4-bit (\texttt{alu\_op})
    \item Έξοδος αποτελέσματος 32-bit (\texttt{result})
    \item Σήμα μηδενικού (\texttt{zero}) που ενεργοποιείται όταν το αποτέλεσμα είναι 0
    \item Σήμα υπερχείλισης (\texttt{ovf}) για αριθμητικές πράξεις
\end{itemize}

\subsection{Ανίχνευση Υπερχείλισης}

Η υπερχείλιση ανιχνεύεται ως εξής:
\begin{itemize}
    \item \textbf{Πρόσθεση:} Όταν τα πρόσημα των τελεστέων είναι ίδια αλλά το πρόσημο του αποτελέσματος διαφέρει
    \item \textbf{Αφαίρεση:} Όταν τα πρόσημα διαφέρουν και το αποτέλεσμα έχει το πρόσημο του αφαιρετέου
    \item \textbf{Πολλαπλασιασμός:} Όταν τα υψηλότερα 33 bits του αποτελέσματος 64-bit δεν είναι επέκταση προσήμου
\end{itemize}

\subsection{Κώδικας - alu.v}

\begin{lstlisting}[caption=Βασική δομή του module ALU]
module alu (
    input  wire [31:0] op1,
    input  wire [31:0] op2,
    input  wire [3:0]  alu_op,
    output wire        zero,
    output reg  [31:0] result,
    output reg         ovf
);
    // Σταθερές λειτουργίας
    parameter [3:0] ALUOP_ADD  = 4'b0100;
    parameter [3:0] ALUOP_SUB  = 4'b0101;
    parameter [3:0] ALUOP_MULT = 4'b0110;
    // ... υπόλοιπες σταθερές
    
    always @(*) begin
        case (alu_op)
            ALUOP_ADD: begin
                result = op1 + op2;
                ovf = (op1[31] == op2[31]) && 
                      (result[31] != op1[31]);
            end
            // ... υπόλοιπες περιπτώσεις
        endcase
    end
    
    assign zero = (result == 32'b0);
endmodule
\end{lstlisting}