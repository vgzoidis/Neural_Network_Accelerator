%=======================================================================
\section{Άσκηση 2: Αριθμομηχανή (Calculator)}
%=======================================================================

\subsection{Περιγραφή}
Σε αυτή την άσκηση σχεδιάστηκε ένα κύκλωμα αριθμομηχανής (calc.v) που χρησιμοποιεί την ALU της Άσκησης 1. Το κύκλωμα διατηρεί μια τρέχουσα τιμή σε έναν συσσωρευτή (accumulator) 16-bit και επιτρέπει στον χρήστη να εκτελεί ορισμένες αριθμητικές και λογικές πράξεις μέσω της ALU. Οι λειτουργίες της αριθμομηχανής επιλέγονται μέσω τριών κουμπιών (btnl, btnr, btnd) και το αποτέλεσμα εμφανίζεται στην έξοδο led (16-bit).

\subsection{Αρχιτεκτονική}
Το calc module υλοποιήθηκε ως ακολουθιακό κύκλωμα με τις ακόλουθες θύρες εισόδου/εξόδου:

\begin{table}[H]
\centering
\caption{Θύρες του module calc}
\begin{tabular}{|l|c|c|p{7cm}|}
\hline
\textbf{Θύρα} & \textbf{Κατεύθυνση} & \textbf{Πλάτος} & \textbf{Περιγραφή} \\
\hline
\texttt{clk} & Είσοδος & 1 & Σήμα ρολογιού \\
\texttt{btnc} & Είσοδος & 1 & Κεντρικό πλήκτρο - εκτέλεση πράξης \\
\texttt{btnac} & Είσοδος & 1 & Πλήκτρο εκκαθάρισης (all clear) \\
\texttt{btnl} & Είσοδος & 1 & Αριστερό πλήκτρο (επιλογής) \\
\texttt{btnr} & Είσοδος & 1 & Δεξί πλήκτρο (επιλογής) \\
\texttt{btnd} & Είσοδος & 1 & Κάτω πλήκτρο (επιλογής) \\
\texttt{sw} & Είσοδος & 16 & Διακόπτες εισαγωγής δεδομένων \\
\hdashline
\texttt{led} & Έξοδος & 16 & LED εξόδου του accumulator \\
\hline
\end{tabular}
\end{table}

\subsubsection{Συσσωρευτής (Accumulator)}

Ο accumulator είναι ένας σύγχρονος εσωτερικός καταχωρητής 16-bit με τις εξής ιδιότητες:
\begin{itemize}
    \item Συνδεδεμένος με την είσοδο ρολογιού (\texttt{clk})
    \item Σύγχρονος μηδενισμός με το πάτημα του \texttt{btnac}
    \item Ενημερώνεται με τα 16 χαμηλότερα bits του αποτελέσματος της ALU όταν πατηθεί το \texttt{btnc}
    \item Η τιμή του εμφανίζεται στα LED
\end{itemize}

\begin{lstlisting}[caption=Συμπεριφορά Καταχωρητή Accumulator, language=Verilog]
always @(posedge clk) begin
    if (btnac)
        accumulator <= 16'b0;            // Synchronous Reset
    else if (btnc)
        accumulator <= alu_result[15:0];
end
\end{lstlisting}

\subsubsection{Επέκταση Προσήμου (Sign Extension)}

Επειδή η ALU δέχεται τελεστές 32-bit και ο accumulator/switches είναι 16-bit, εφαρμόζεται επέκταση προσήμου (sign extension). Το MSB (bit 15) επαναλαμβάνεται 16 φορές για τα υψηλότερα bits:

\begin{lstlisting}[caption=Επέκταση προσήμου του MSB, language=Verilog]
assign op1_extended={{16{accumulator[15]}},accumulator}; //acc -> 32-bit
assign op2_extended={{16{sw[15]}},sw};                   // sw -> 32-bit
\end{lstlisting}

\subsection{Encoder (calc\_enc.v)}
Ο encoder (calc\_enc.v) υλοποιήθηκε σε \textbf{structural Verilog} χρησιμοποιώντας βασικές πύλες (AND, OR, NOT, XOR) σύμφωνα με τα Σχήματα 2-5 της εκφώνησης. Οι λογικές εξισώσεις για κάθε bit του alu\_op είναι:

\begin{align}
\text{(Σχ. 2):}\quad\text{alu\_op}[0] &= (\overline{\text{btnl}} \cdot \text{btnd}) + ((\text{btnl} \cdot \text{btnr}) \cdot \overline{\text{btnd}}) \\
\text{(Σχ. 3):}\quad\text{alu\_op}[1] &= \text{btnl} \cdot (\overline{\text{btnr}} + \overline{\text{btnd}}) \\
\text{(Σχ. 4):}\quad\text{alu\_op}[2] &= (\overline{\text{btnl}} \cdot \text{btnr}) + (\text{btnl} \cdot \overline{(\text{btnr} \oplus \text{btnd})}) \\
\text{(Σχ. 5):}\quad\text{alu\_op}[3] &= (\text{btnl} \cdot \text{btnr}) + (\text{btnl} \cdot \text{btnd})
\end{align}

\justify
Ο πίνακας αλήθειας που προκύπτει από αυτές τις εξισώσεις:

\begin{center}
\begin{tabular}{|c|c|c|c|c|}
\hline
btnl & btnr & btnd & alu\_op & Λειτουργία \\
\hline
0 & 0 & 0 & 0000 & SRL\\
0 & 0 & 1 & 0001 & SLL\\
\hdashline
0 & 1 & 0 & 0100 & ADD\\
0 & 1 & 1 & 0101 & SUB\\
1 & 0 & 0 & 0110 & MULT\\
\hdashline
1 & 0 & 1 & 1010 & NOR \\
1 & 1 & 0 & 1011 & NAND \\
1 & 1 & 1 & 1100 & XOR \\
\hline
\end{tabular}
\end{center}

\subsection{Αποτελέσματα Testbench}
Η καλή λειτουργία της αριθμομηχανής επιβεβαιώθηκε αποτυπώνοντας τον έλεγχο της εκφώνησης στο calc\_tb.v:

\begin{table}[H]
    \centering
    \caption{Αποτελέσματα Προσομοίωσης (Calculator Testbench)}
    \label{tab:calc_testbench_results}
    \begin{tabular}{|c|c|c|c|c|c|c|}
        \hline
        \textbf{\#} & \textbf{Input} & \textbf{Switches} & \textbf{Function} & \textbf{Expected} & \textbf{\hspace{0.2cm}Actual\hspace{0.2cm}} & \textbf{Test} \\
         & \footnotesize{(btnl,btnr,btnd)} & (input) & (in ALU) & \textbf{Result} & \textbf{Result} & \textbf{Result}\\
        \hline
        1 & btnac & \texttt{xxxx} & Reset & \texttt{0x0000} & \texttt{0x0000} & {\textbf{PASS}} \\
        2 & 0, 1, 0 & \texttt{0x285a} & ADD & \texttt{0x285a} & \texttt{0x285a} & {\textbf{PASS}} \\
        3 & 1, 1, 1 & \texttt{0x04c8} & XOR & \texttt{0x2c92} & \texttt{0x2c92} & {\textbf{PASS}} \\
        4 & 0, 0, 0 & \texttt{0x0005} & SRL & \texttt{0x0164} & \texttt{0x0164} & {\textbf{PASS}} \\
        5 & 1, 0, 1 & \texttt{0xa085} & NOR & \texttt{0x5e1a} & \texttt{0x5e1a} & {\textbf{PASS}} \\
        6 & 1, 0, 0 & \texttt{0x07fe} & MULT & \texttt{0x13cc} & \texttt{0x13cc} & {\textbf{PASS}} \\
        7 & 0, 0, 1 & \texttt{0x0004} & SLL & \texttt{0x3cc0} & \texttt{0x3cc0} & {\textbf{PASS}} \\
        8 & 1, 1, 0 & \texttt{0xfa65} & NAND & \texttt{0xc7bf} & \texttt{0xc7bf} & {\textbf{PASS}} \\
        9 & 0, 1, 1 & \texttt{0xb2e4} & SUB & \texttt{0x14db} & \texttt{0x14db} & {\textbf{PASS}} \\
        \hline
        \multicolumn{7}{|c|}{\textbf{Συνολικό Αποτέλεσμα: \textcolor{green!50!black}{9/9 PASS}}} \\
        \hline
    \end{tabular}
\end{table}

\subsection{Κυματομορφές Προσομοίωσης}

Τα παραπάνω μπορούμε να διαπιστώσουμε και στις κυματομορφές από την προσομοίωση του testbench (Figure \ref{fig:waveform_ex2}). Παρατηρούμε:
\begin{itemize}
    \item Η τιμή της εξόδου (\texttt{led}) ενημερώνεται -σύγχρονα- στην ανερχόμενη ακμή του \texttt{clk} όταν το \texttt{btnc} είναι ενεργό.
    \item Το \texttt{btnac} λειτουργεί ως reset πάλι σύγχρονα.
    \item Οι διαδοχικές τιμές αντιστοιχούν στα αναμενόμενα αποτελέσματα: \texttt{0x0} $\rightarrow$ \texttt{0x285a} $\rightarrow$ \texttt{0x2c92} $\rightarrow$ \texttt{0x0164} $\rightarrow$ \texttt{0x5e1a} $\rightarrow$ \texttt{0x13cc} $\rightarrow$ \texttt{0x3cc0} $\rightarrow$ \texttt{0xc7bf} $\rightarrow$ \texttt{0x14db}
\end{itemize}

\begin{figure}[htbp]
    \centering
    \makebox[\textwidth][c]{\includegraphics[width=1.1\textwidth]{Waveforms/Exercise_2 - Waveform.png}}
    \caption{Κυματομορφές προσομοίωσης της αριθμομηχανής. Φαίνεται η εξέλιξη της τιμής του led μετά από κάθε πράξη.}
    \label{fig:waveform_ex2}
\end{figure}
\newpage