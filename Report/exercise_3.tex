%=======================================================================
\section{Άσκηση 3: Register File}
%=======================================================================

\subsection{Περιγραφή}
Το αρχείο καταχωρητών (Register File) είναι ένα σημαντικό στοιχείο κάθε ψηφιακού συστήματος. Σε αυτή την άσκηση υλοποιήθηκε ένα Register File 16 καταχωρητών, το οποίο θα χρησιμοποιηθεί στον AI accelerator (Άσκηση 4) για την αποθήκευση των βαρών (weights) και των πολώσεων (biases) του νευρωνικού δικτύου. Το Register File που υλοποιήθηκε υποστηρίζει παραμετροποιήσιμο πλάτος δεδομένων (DATAWIDTH) και ασύγχρονο reset.


\subsection{Αρχιτεκτονική}
Το regfile module σχεδιάστηκε ως ακολουθιακό κύκλωμα με τις ακόλουθες θύρες εισόδου/εξόδου:
\begin{table}[H]
\caption{Θύρες του module regfile}
\begin{tabular}{|l|c|c|p{7cm}|}
\hline
\textbf{Θύρα} & \textbf{Κατεύθυνση} & \textbf{Πλάτος} & \textbf{Περιγραφή} \\
\hline
\texttt{clk} & Είσοδος & 1 & Σήμα ρολογιού \\
\texttt{resetn} & Είσοδος & 1 & Σήμα επαναφοράς (active low) \\
\texttt{write} & Είσοδος & 1 & Σήμα ενεργοποίησης εγγραφής \\
\texttt{readReg1-4} & Είσοδοι & 4 & Διευθύνσεις ανάγνωσης (4 θύρες) \\
\texttt{writeReg1-2} & Είσοδοι & 4 & Διευθύνσεις εγγραφής (2 θύρες) \\
\texttt{writeData1-2} & Είσοδοι & DW & Δεδομένα προς εγγραφή \\
\hdashline
\texttt{readData1-4} & Έξοδοι & DW & Δεδομένα ανάγνωσης (4 θύρες) \\
\hline
\end{tabular}
\end{table}

\subsection{Λειτουργία Reset και Read/Write}
Η λειτουργία του Register File βασίζεται σε ένα κεντρικό \texttt{always} block.
\begin{itemize}
    \item \textbf{Reset}: Όλοι οι καταχωρητές αρχικοποιούνται \textbf{ασύγχρονα} στο μηδέν όταν το \texttt{resetn} είναι χαμηλό.
    \item \textbf{Read/Write}: Tα δεδομένα διαβάζονται/εγγράφονται στους επιλεγμένους καταχωρητές \textbf{σύγχρονα} με την ανερχόμενη ακμή του clk.
\end{itemize}

\begin{lstlisting} [caption=Always block του Register file, language=Verilog]
always @(posedge clk or negedge resetn) begin
    if (!resetn) begin // Asynchronous reset
        for (i = 0; i < 16; i = i + 1) begin
            registers[i] <= {DATAWIDTH{1'b0}}; // Clear all
        end
        readData1 <= {DATAWIDTH{1'b0}};
        ...
    end
    else begin // Synchronous read/write
        if (write) begin
            registers[writeReg1] <= writeData1; // Write
            ...
        end
    end
\end{lstlisting}

\subsection{Data Forwarding (Προώθηση Δεδομένων)}
Ένα κρίσιμο χαρακτηριστικό της αρχιτεκτονικής είναι η λογική Data Forwarding. Σε περίπτωση που η διεύθυνση ανάγνωσης ταυτίζεται με μια διεύθυνση εγγραφής στον ίδιο κύκλο, πρέπει να δοθεί \textbf{προτεραιότητα στην εγγραφή}.

Αντί να διαβαστεί η "παλιά" τιμή του καταχωρητή, στην έξοδο οδηγείται απευθείας η νέα τιμή (\texttt{writeData}). Η προτεραιότητα καθορίζεται ως εξής:

\begin{enumerate}
    \item Forwarding από Write Port 1 (Υψηλότερη προτεραιότητα)
    \item Forwarding από Write Port 2 στους επιλεγμένους καταχωρητές \item Ανάγνωση από Register Array (Κανονική λειτουργία)
\end{enumerate}

\begin{lstlisting}[caption=Υλοποίηση Data Forwarding, language=Verilog]
// Example for Port 1
if (readReg1 == writeReg1)      
    readData1 <= writeData1;    // Priority 1
else if (readReg1 == writeReg2) 
    readData1 <= writeData2;    // Priority 2
else                            
    readData1 <= registers[readReg1]; // Normal Read
\end{lstlisting}
\newpage