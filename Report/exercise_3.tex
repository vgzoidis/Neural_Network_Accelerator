%==============================================================================
\section{Άσκηση 3: Register File}
%==============================================================================

\subsection{Περιγραφή}

Το Register File υλοποιεί:
\begin{itemize}
    \item 16 καταχωρητές × 32 bits (παραμετροποιήσιμο DATAWIDTH)
    \item 4 θύρες ανάγνωσης (readData1-4)
    \item 2 θύρες εγγραφής (writeData1-2)
    \item Ασύγχρονο reset (active low)
    \item Data forwarding για αποφυγή hazards
\end{itemize}

\subsection{Data Forwarding}

Σε περίπτωση που η διεύθυνση ανάγνωσης ταυτίζεται με διεύθυνση εγγραφής και το \texttt{write} είναι ενεργό, επιστρέφονται απευθείας τα νέα δεδομένα (write-first behavior):

\begin{lstlisting}[caption=Data Forwarding με προτεραιότητα εγγραφής]
if (write) begin
    // Εγγραφή δεδομένων
    registers[writeReg1] <= writeData1;
    registers[writeReg2] <= writeData2;
    
    // Forwarding: προτεραιότητα στο writeData
    if (readReg1 == writeReg1)      readData1 <= writeData1;
    else if (readReg1 == writeReg2) readData1 <= writeData2;
    else                            readData1 <= registers[readReg1];
    // ... ομοίως για ports 2, 3, 4
end
else begin
    // Μόνο ανάγνωση
    readData1 <= registers[readReg1];
    // ...
end
\end{lstlisting}

\subsection{Σχεδιαστικές Αποφάσεις - Σύγχρονη Ανάγνωση}

Σύμφωνα με την εκφώνηση, το Register File υλοποιεί \textbf{σύγχρονη ανάγνωση} μέσα σε ένα always block:

\begin{itemize}
    \item \textbf{Απαίτηση εκφώνησης:} ``Μέσω ενός always block, θα γίνεται η εγγραφή ή ανάγνωση''
    \item \textbf{Αμοιβαίος αποκλεισμός:} ``θεωρούμε ότι δεν μπορούμε να πραγματοποιήσουμε ταυτόχρονα και ανάγνωση και εγγραφή''
    \item \textbf{Registered outputs:} Τα \texttt{readData} ενημερώνονται στο \texttt{posedge clk}
    \item \textbf{Επίπτωση στο FSM:} Τα δεδομένα είναι διαθέσιμα στον \textbf{επόμενο} κύκλο
\end{itemize}

\subsubsection{Συμβατότητα με το Νευρωνικό Δίκτυο - Τεχνική Pre-fetching}

\textbf{Απαίτηση της εκφώνησης:} Το Register File υλοποιήθηκε με \textbf{σύγχρονη ανάγνωση} (ένα always block για εγγραφή ή ανάγνωση), σύμφωνα με τις οδηγίες:
\begin{quote}
\textit{``Μέσω ενός always block, θα γίνεται η εγγραφή ή ανάγνωση''}
\end{quote}

Αυτή η σχεδιαστική επιλογή σημαίνει ότι τα δεδομένα εμφανίζονται στις εξόδους \texttt{readData} \textbf{έναν κύκλο μετά} τον ορισμό της διεύθυνσης \texttt{readReg}.

\textbf{Τεχνική Pre-fetching:} Για να αντιμετωπιστεί αυτή η καθυστέρηση, το FSM του \texttt{nn.v} εφαρμόζει τεχνική \textbf{pre-fetching διευθύνσεων}: οι διευθύνσεις τίθενται \textbf{ένα κύκλο νωρίτερα} από ότι χρειάζονται τα δεδομένα.

\begin{table}[H]
\centering
\caption{Pre-fetching διευθύνσεων RegFile στο FSM}
\begin{tabular}{|l|l|l|}
\hline
\textbf{Τρέχουσα Κατάσταση} & \textbf{Διευθύνσεις που θέτουμε (pre-fetch)} & \textbf{Δεδομένα διαθέσιμα στην} \\
\hline
S\_IDLE / S\_DEACTIVATED & shift\_bias\_1, shift\_bias\_2 & S\_PREPROCESS \\
S\_PREPROCESS & weight\_1, bias\_1, weight\_2, bias\_2 & S\_INPUT\_LAYER \\
S\_INPUT\_LAYER & weight\_3, weight\_4, bias\_3 & S\_OUTPUT\_LAYER1 \\
S\_OUTPUT\_LAYER1 & shift\_bias\_3 & S\_POSTPROCESS \\
\hline
\end{tabular}
\end{table}

Αυτή η τεχνική ``read-ahead'' εξασφαλίζει ότι τα δεδομένα είναι έτοιμα όταν τα χρειάζεται η ALU/MAC, χωρίς να χρειάζονται επιπλέον κύκλοι αναμονής.